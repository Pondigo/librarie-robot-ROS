%%%%%%%%%%%%%%%%%%%%%%%%%%%%%%%%%%%%%%%%
% Tipo de documento y carga de paquetes,
 % NO es necesario modificar
%%%%%%%%%%%%%%%%%%%%%%%%%%%%%%%%%%%%%%%%%%
\documentclass[12pt, a4paper]{article}

\usepackage{graphicx}
\usepackage{geometry}
\usepackage{multicol}
\graphicspath{{Imagenes/}}
\geometry{ %Definición del tamaño de los márgenes
a4paper,
left=10mm,
right=10mm,
top=10mm,
bottom=15mm}

\font\titlefont=cmr12 at 40pt

%%%%%%%%%%%%%%%%%%%%%%%%%%%%%%%
%%%%% Datos para el título %%%%
%%%%%%%%%%%%%%%%%%%%%%%%%%%%%%
\title{
	\includegraphics[width=10cm]{udlap.eps}\vspace{2ex} 
	\\ 
	\titlefont Minuta \#1\vspace{-3ex}
	\author{} % No agregar autor, esta linea es para evitar un warning
}
\date{
\huge 01 febrero 2022 \\ \vspace{0.5ex}
\large Inicio: 19:30 H \quad Término: 20:00 H}
%%%%%%%%%%%%%%%%%%%%%%%%%%%%%%%%%%%%%%%%%
%%%%%% Inicio del documento  %%%%%%%%%%%
%%%%%%%%%%%%%%%%%%%%%%%%%%%%%%%%%%%%%%%%
\begin{document}
% Toma de asistencia
\maketitle %Generar título con los datos previos 
% ID y nombre de los estudiantes presentes
\begin{minipage}[t]{0.4\textwidth}
\section*{Presentes}
\begin{itemize}
\item 163044 - Luis Fernando Rosas Ordaz
\item 164299 - Lenora Pondigo Santamar\'ia 
\item 164321 - Luis Alberto Saucedo L\'opez 
\end{itemize}
\end{minipage}
% ID y nombre de los estudiantes ausentes
\begin{minipage}[t]{0.5\textwidth}
\section*{Ausentes}
\begin{itemize}
\item No aplica
\end{itemize}
\end{minipage}

%%%%%%%%%%%%%%%%%%%
% Agenda de la reunión
%%%%%%%%%%%%%%%%%%%
\section*{Agenda}
Los temas discutidos en la reuni\'on fueron los siguientes:.
\begin{enumerate}
\item Metodología a utilizar
\item Fortalezas y debilidades del equipo
\item Encargados de las actividades
\item Selecci\'on de proyecto
\end{enumerate}
% Discusión "detallada" de los temas de la reunión.
\section{Metodología a utilizar}
La metodología que se empleara sera Scrum, debido a que es una metodología que permite una mayor agilidad en el desarrollo de proyectos.
De igual forma, para este proyecto esta metolog\'ia es la ideal, ya que optimizara el tiempo en el desarrollo e implementaci\'on del robot ordenador de libros
\section{Fortalezas y debilidades del equipo}
Las fortalezas y debilidades de los miembros del equipo son:
\begin{itemize}
	\item Lenora Pondigo 
	\begin{itemize}
	    \item Fortalezas: Programaci\'on, control de versiones con git y dibujo 3D
	    \item Debilidades: Organizaci\'on y procrastinaci\'on
    \end{itemize}
    \item Luis Fernando Rosas 
	\begin{itemize}
	    \item Fortalezas: Dibujo 3D, organizaci\'on y liderazgo
	    \item Debilidades: Indeciso y poco paciente 
    \end{itemize}
    \item Luis Alberto Saucedo 
	\begin{itemize}
	    \item Fortalezas: Dedicaci\'on, organizaci\'on y trabajo en equipo
	    \item Debilidades: Procastinaci\'on y confiado
    \end{itemize}
\end{itemize}
\section{Encargados de las actividades}
Las cuatro principales actividades del proyecto, estaran supervisadas por un miembro del equipo, de la siguiente forma: 
\begin{itemize}
	\item Administraci\'on de proyecto: Luis Alberto Saucedo
	\item Diseño CAD: Luis Fernando Rosas
	\item Sistemas de control y programaci\'on: Lenora Pondigo
	\item Implementaci\'on: Todos
\end{itemize}
Es importante señalar que, la actividad de implementaci\'on sera revisado por todos los integrantes, ya que esta actividad requiere las habilidades de todos para una implementaci\'on funcional. 
\section{Selecci\'on de proyecto}
El proyecto que se va a desarrollar es el robot ordenador de libros. Este robot fue seleccionado porque se tienen los conocimientos suficientes para desarrollarlo e implementarlo de una forma eficiente.
%%%%%%%%%%%%%%%%%%%
% Tareas a realizar
%%%%%%%%%%%%%%%%%%%
\section*{Tasks}
Las tareas pendientes para la elaboración del proyecto son:
\setcounter{section}{0}

\section{Planeaci\'on de la metolog\'ia} %Copiar y pegar más secciones si necesario

Se realizar\'a la primera planeaci\'on Scrum, esto con la finalidad de optimizar la elaboraci\'on del proyecto

\begin{itemize}
    \item Encargado: Luis Saucedo
\end{itemize}

\section{Analis\'is de diseño} %Copiar y pegar más secciones si necesario

Se realizar\'a el primer anal\'is del diseño del robot ordenador de libros. Esto con la finalidad de ir conceptualizando que es lo que se necesita y las especificaciones necesarias a cubrir.

\begin{itemize}
    \item Encargado: Luis Rosas y Lenora Pondigo
\end{itemize}

\section*{Future Meetings} %Próxia reunión, al menos una reunión semanal
La siguiente reuni\'on se va a realizar el d\'ia 10 de febrero del 2022, entre el horario de 16:00 H a 17:40 H
\end{document}
% Código modificado de la fuente original, disponible en 
%{http://github.com/cmichi/latex-template-collection}